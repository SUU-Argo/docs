\documentclass{article}
\usepackage[utf8]{inputenc}
\usepackage{polski}
\usepackage{minted} % code snippets
\usepackage{geometry} % margins
\geometry{
    left=30mm,
    top=25mm,
    right=30mm
}
\setlength{\parindent}{0cm}
\usepackage{graphicx} % images
\usepackage{hyperref} % links
\usepackage{titling} % center front page
\usepackage[T1]{fontenc} % fix polish characters bug
\usepackage{textcomp} % no
\usepackage[dvipsnames]{xcolor} % colors

\usepackage{tabularx} % table
\newcolumntype{A}{>{\hsize=.05\hsize}X}
\newcolumntype{B}{>{\hsize=.10\hsize}X}
\newcolumntype{C}{>{\hsize=.15\hsize}X}
\newcolumntype{D}{>{\hsize=.20\hsize}X}
\newcolumntype{E}{>{\hsize=.25\hsize}X}
\newcolumntype{F}{>{\hsize=.30\hsize}X}
\newcolumntype{G}{>{\hsize=.35\hsize}X}
\newcolumntype{H}{>{\hsize=.40\hsize}X}
\newcolumntype{I}{>{\hsize=.45\hsize}X}
\newcolumntype{J}{>{\hsize=.50\hsize}X}
\newcolumntype{K}{>{\hsize=.55\hsize}X}
\newcolumntype{L}{>{\hsize=.60\hsize}X}
\newcolumntype{M}{>{\hsize=.65\hsize}X}
\newcolumntype{N}{>{\hsize=.70\hsize}X}
\newcolumntype{O}{>{\hsize=.75\hsize}X}
\newcolumntype{P}{>{\hsize=.80\hsize}X}
\newcolumntype{Q}{>{\hsize=.85\hsize}X}
\newcolumntype{R}{>{\hsize=.90\hsize}X}
\newcolumntype{S}{>{\hsize=.95\hsize}X}
\newcolumntype{T}{>{\hsize=1.00\hsize}X}

\renewcommand\maketitlehooka{\null\mbox{}\vfill}
\renewcommand\maketitlehookd{\vfill\null}

% images
\usepackage{float}
\usepackage{graphicx}
\newcommand{\img}[4]{
    \begin{figure}
    \centering
    \includegraphics[width=#4\textwidth]{#1}
    \caption{#2}
    \label{#3}
    \end{figure}
}

% bibliography
\usepackage[
style=numeric,
sorting=none,
isbn=false,
doi=true,
url=true,
backref=false,
backrefstyle=none,
maxnames=10,
giveninits=true,
abbreviate=true,
defernumbers=false,
backend=biber]{biblatex}
\addbibresource{bibliography.bib}

%%%%%%%%%%%%%%%%%%%%%%%%%%%%%%%%%%%%%%%%%%%%%%%%%%%%%%%%%%%%%%%%%%%%%%%%%%%%%%%%%%

\title{\textbf{Argo}}
\author{Gabriel Kaźmierczak\\ Dariusz Piwowarski\\ Wojciech Przybytek\\ Przemysław Roman}
\date{2024}

\begin{document}
\begin{titlingpage}
\maketitle
\end{titlingpage}


\newpage
\tableofcontents


\newpage
\section{Wstęp}
% Ogólny opis tematyki

\subsection*{Podział pracy w zespole}
Ponieważ w ramach tego projektu omawiamy dwa zastosowania Argo zdecydowaliśmy na podział zespołu na dwa podzespoły, każdy omawiający jedno z zastosowań.
Docelowo praca obu podzespołów ma zostać połączona, tworząc integralną całość.

\subsubsection*{Podzespół 1 - Argo CD}
Osoby: Dariusz Piwowarski, Przemysław Roman

Cele: Stworzenie klastra Kubernetes na AWS oraz uruchomienie w nim Argo CD i zintegrowanie go z repozytorium aplikacji

\subsubsection*{Podzespół 2 - Argo Workflo}
Osoby: Gabriel Kaźmierczak, Wojciech Przybytek

Cele: Zapoznanie się z API Argo Workflows, napisanie aplikacji wykorzystującej to API oraz udostępniającej własny interfejs


\newpage
\section{Podstawy teoretyczne}
\begin{description}
\item{\textbf{Kubernetes}\cite{kubernetes}} to oprogramowanie open source służące do automatyzacji wdrażania, skalowania i zarządzania aplikacjami kontenerowymi. Zostało zaprojektowane przez Google, a obecnie jest utrzymywane przez Cloud Native Computing Foundation. Kubernetes dostarcza platformę do uruchamiania aplikacji w kontenerach na dużą skalę, zarządzając infrastrukturą i zapewniając funkcje, takie jak self-healing (automatyczne restartowanie kontenerów), odkrywanie usług i load balancing, przechowywanie danych, skalowanie i wiele innych.

\item{\textbf{Argo}\cite{argo}} jest projektem open source, który dostarcza narzędzia do tworzenia, uruchamiania i zarządzania pracami w kontenerach Kubernetes. Składa się z podprojektów, w których skład wchodzą Argo Workflows, Argo CD, Argo Rollouts oraz Argo Events.

\item{\textbf{Argo Workflows}\cite{argo-workflows}} to silnik do tworzenia i uruchamiania prac w kontenerach. Pozwala na definiowanie złożonych przepływów pracy (workflow), które mogą obejmować wiele zadań uruchamianych w różnych kontenerach, z możliwością kontroli przepływu, takiej jak równoległe lub sekwencyjne uruchamianie zadań, decyzje warunkowe, pętle i inne. Argo Workflows jest szczególnie przydatne w środowiskach takich jak przetwarzanie danych, czy uczenie maszynowe, gdzie istnieje potrzeba koordynacji wielu zadań.

\item{\textbf{Argo CD}\cite{argo-cd}} to narzędzie do ciągłej dostawy (Continuous Delivery, CD), które implementuje podejście GitOps do zarządzania infrastrukturą. W modelu GitOps, żądane stany systemów są zapisywane w repozytorium Git, a narzędzia automatycznie aktualizują systemy, aby dopasować je do stanu zapisanego w Git. Argo CD monitoruje repozytorium i automatycznie wdraża zmiany na środowiska Kubernetes, gdy stan w Git się zmienia.
\end{description}

Oba omawiane podprojekty Argo są zintegrowane z Kubernetes i wykorzystują jego funkcje, takie jak API, schematy autoryzacji i mechanizmy skalowania. Dzięki temu są one naturalnym rozszerzeniem ekosystemu Kubernetes i mogą być łatwo zintegrowane z innymi narzędziami w tym ekosystemie.


\section{Stos technologiczny}
\begin{description}
\item{\textbf{Amazon Elastic Kubernetes Service}\cite{aws-eks}} zapewni infrastrukturę na której będzie działał klaster Kubernetes

\item{\textbf{GitHub}\cite{github}} umożliwi przechowywanie repozytorium Git z kodem źródłowym aplikacji

\item{\textbf{Python 3}\cite{python}} jako język implementacji aplikacji

\item{\textbf{Hera}\cite{hera}} to biblioteka, która zostanie wykorzystane do komunikacji z API Argo
\end{description}


\newpage
\section{Koncepcja studium przypadku}
Demo będzie zawierać aplikację działającą w klastrze Kubernetes, która zaprezentuje możliwości narzędzi Argo Workflows oraz Argo CD.

Aplikacja będzie służyć jako punkt wejścia do uruchamiania zadań w Argo Workflows. Użytkownik będzie mógł wysyłać
żądania HTTP do aplikacji uruchamiające odpowiednie workflowy za pomocą API do Argo Workflows.

Każda zmiana w repozytorium przechowującym kod aplikacji będzie automatycznie wykrywana przez Argo CD, które następnie zaktualizuje działającą aplikację na klastrze.


\newpage
\section{Architektura rozwiązania}


\newpage
\section{Konfiguracja środowiska}
% opis słowny

\img{img/suu_env_conf.drawio.png}{Konfiguracja środowiska}{fig:env-conf}{1}


\newpage
\section{Instalacja narzędzi}


\newpage
\section{Reprodukcja środowiska}


\newpage
\section{Reprodukcja demo}


\newpage
\section{Podsumowanie}


\newpage
\nocite{*}
\printbibliography[heading=bibintoc,title={Źródła}]

\end{document}
