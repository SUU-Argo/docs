\documentclass{article}
\usepackage[utf8]{inputenc}
\usepackage{polski}
\usepackage{minted}% code snippets
\usepackage{geometry} % margins
\geometry{
    left=30mm,
    top=25mm,
    right=30mm
}
\setlength{\parindent}{0cm}
\usepackage{graphicx} % images
\usepackage{hyperref} % links
\usepackage{titling} % center front page
\usepackage[T1]{fontenc} % fix polish characters bug
\usepackage{textcomp} % no
\usepackage[dvipsnames]{xcolor} % colors

\usepackage{tabularx} % table
\newcolumntype{A}{>{\hsize=.05\hsize}X}
\newcolumntype{B}{>{\hsize=.10\hsize}X}
\newcolumntype{C}{>{\hsize=.15\hsize}X}
\newcolumntype{D}{>{\hsize=.20\hsize}X}
\newcolumntype{E}{>{\hsize=.25\hsize}X}
\newcolumntype{F}{>{\hsize=.30\hsize}X}
\newcolumntype{G}{>{\hsize=.35\hsize}X}
\newcolumntype{H}{>{\hsize=.40\hsize}X}
\newcolumntype{I}{>{\hsize=.45\hsize}X}
\newcolumntype{J}{>{\hsize=.50\hsize}X}
\newcolumntype{K}{>{\hsize=.55\hsize}X}
\newcolumntype{L}{>{\hsize=.60\hsize}X}
\newcolumntype{M}{>{\hsize=.65\hsize}X}
\newcolumntype{N}{>{\hsize=.70\hsize}X}
\newcolumntype{O}{>{\hsize=.75\hsize}X}
\newcolumntype{P}{>{\hsize=.80\hsize}X}
\newcolumntype{Q}{>{\hsize=.85\hsize}X}
\newcolumntype{R}{>{\hsize=.90\hsize}X}
\newcolumntype{S}{>{\hsize=.95\hsize}X}
\newcolumntype{T}{>{\hsize=1.00\hsize}X}

\renewcommand\maketitlehooka{\null\mbox{}\vfill}
\renewcommand\maketitlehookd{\vfill\null}

% images
\usepackage{float}
\usepackage{graphicx}
\newcommand{\img}[4]{
    \begin{figure}[!htbp]
    \centering
    \includegraphics[width=#4\textwidth]{#1}
    \caption{#2}
    \label{#3}
    \end{figure}
}

% bibliography
\usepackage[
style=numeric,
sorting=none,
isbn=false,
doi=true,
url=true,
backref=false,
backrefstyle=none,
maxnames=10,
giveninits=true,
abbreviate=true,
defernumbers=false,
backend=biber]{biblatex}
\addbibresource{bibliography.bib}

% \AtEveryBibitem{
%     \clearfield{urlyear}
%     \clearfield{urlmonth}
% }

% md like code snippet (`)
\usepackage{tcolorbox}
\newcommand{\code}[1]{\mbox{
    \ttfamily
    \tcbox[
        on line,
        boxsep=0pt, left=4pt, right=4pt, top=2pt, bottom=2pt,
        toprule=4pt, rightrule=0pt, bottomrule=0pt, leftrule=-4pt,
        oversize=0pt, enlarge left by=0pt, enlarge right by=0pt,
        colframe=white, colback=black!12
    ]{#1}
}}
% permits newlines, add a newline before (```)
\newcommand{\codeblock}[1]{\mbox{
    \ttfamily
    \tcbox[
        on line,
        boxsep=0pt, left=4pt, right=4pt, top=2pt, bottom=2pt,
        toprule=4pt, rightrule=0pt, bottomrule=0pt, leftrule=-4pt,
        oversize=0pt, enlarge left by=0pt, enlarge right by=0pt,
        colframe=white, colback=black!12, capture=minipage
    ]{#1}
}}

\renewcommand{\figureautorefname}{Rys.}
\renewcommand{\tableautorefname}{Tab.}
\renewcommand{\sectionautorefname}{Sek.}
\renewcommand{\subsectionautorefname}{Subsek.}

%%%%%%%%%%%%%%%%%%%%%%%%%%%%%%%%%%%%%%%%%%%%%%%%%%%%%%%%%%%%%%%%%%%%%%%%%%%%%%%%%%

\title{\textbf{Środowiska Udostępniania Usług - Argo}}
\author{Gabriel Kaźmierczak\\ Dariusz Piwowarski\\ Wojciech Przybytek\\ Przemysław Roman}
\date{2024}

\begin{document}
\begin{titlingpage}
\maketitle
\end{titlingpage}


\newpage
\tableofcontents


\newpage
\section{Wstęp}
Celem projektu jest zaprezentowanie działania Argo, czyli zestawu open source'owych narzędzi stworzonych do zarządzania zadaniami w kontenerach Kubernetes. Narzędzia te są szeroko stosowane do automatyzacji wdrażania aplikacji, zarządzania ich cyklem życia oraz koordynacji złożonych przepływów pracy. W ramach projektu skupimy się na dwóch kluczowych komponentach Argo: Argo CD, które umożliwia ciągłą dostawę aplikacji przy użyciu podejścia GitOps, oraz Argo Workflows, które pozwala na definiowanie i zarządzanie złożonymi przepływami pracy w środowisku Kubernetes.

\subsection*{Podział pracy w zespole}
Ponieważ w ramach tego projektu omawiamy dwa zastosowania Argo zdecydowaliśmy na podział zespołu na dwa podzespoły, każdy omawiający jedno z zastosowań.
Docelowo praca obu podzespołów ma zostać połączona, tworząc integralną całość.

\subsubsection*{Podzespół 1 - Argo CD}
\textbf{Osoby}: Dariusz Piwowarski, Przemysław Roman \\
\textbf{Cele}: Stworzenie klastra Kubernetes na AWS oraz uruchomienie w nim Argo CD i zintegrowanie go z repozytorium aplikacji

\subsubsection*{Podzespół 2 - Argo Workflows}
\textbf{Osoby}: Gabriel Kaźmierczak, Wojciech Przybytek \\
\textbf{Cele}: Zapoznanie się z API Argo Workflows, napisanie aplikacji wykorzystującej to API oraz udostępniającej własny interfejs

\section{Podstawy teoretyczne}
\begin{description}
\item{\textbf{Kubernetes}\cite{kubernetes}} to oprogramowanie open source służące do automatyzacji wdrażania, skalowania i zarządzania aplikacjami kontenerowymi. Zostało zaprojektowane przez Google, a obecnie jest utrzymywane przez Cloud Native Computing Foundation. Kubernetes dostarcza platformę do uruchamiania aplikacji w kontenerach na dużą skalę, zarządzając infrastrukturą i zapewniając funkcje, takie jak self-healing (automatyczne restartowanie kontenerów), odkrywanie usług i load balancing, przechowywanie danych, skalowanie i wiele innych.

\item{\textbf{Argo}\cite{argo}} jest projektem open source, który dostarcza narzędzia do tworzenia, uruchamiania i zarządzania pracami w kontenerach Kubernetes. Składa się z podprojektów, w których skład wchodzą Argo Workflows, Argo CD, Argo Rollouts oraz Argo Events.

\item{\textbf{Argo Workflows}\cite{argo-workflows}} to silnik do tworzenia i uruchamiania prac w kontenerach. Pozwala na definiowanie złożonych przepływów pracy (workflow), które mogą obejmować wiele zadań uruchamianych w różnych kontenerach, z możliwością kontroli przepływu, takiej jak równoległe lub sekwencyjne uruchamianie zadań, decyzje warunkowe, pętle i inne. Argo Workflows jest szczególnie przydatne w środowiskach takich jak przetwarzanie danych, czy uczenie maszynowe, gdzie istnieje potrzeba koordynacji wielu zadań.

\item{\textbf{Argo CD}\cite{argo-cd}} to narzędzie do ciągłej dostawy (Continuous Delivery, CD), które implementuje podejście GitOps do zarządzania infrastrukturą. W modelu GitOps, żądane stany systemów są zapisywane w repozytorium Git, a narzędzia automatycznie aktualizują systemy, aby dopasować je do stanu zapisanego w Git. Argo CD monitoruje repozytorium i automatycznie wdraża zmiany na środowiska Kubernetes, gdy stan w Git się zmienia.
\end{description}

Oba omawiane podprojekty Argo są zintegrowane z Kubernetes i wykorzystują jego funkcje, takie jak API, schematy autoryzacji i mechanizmy skalowania. Dzięki temu są one naturalnym rozszerzeniem ekosystemu Kubernetes i mogą być łatwo zintegrowane z innymi narzędziami w tym ekosystemie.


\section{Stos technologiczny}
\begin{description}
\item{\textbf{Amazon Elastic Kubernetes Service}\cite{aws-eks}} zapewni infrastrukturę na której będzie działał klaster Kubernetes

\item{\textbf{GitHub}\cite{github}} umożliwi przechowywanie repozytorium Git z kodem źródłowym aplikacji

\item{\textbf{Python 3}\cite{python}} jako język implementacji aplikacji

\item{\textbf{Hera}\cite{hera}} to biblioteka, która zostanie wykorzystane do komunikacji z API Argo
\end{description}


\section{Koncepcja studium przypadku}
Demo będzie zawierać aplikację działającą w klastrze Kubernetes, która zaprezentuje możliwości narzędzi Argo Workflows oraz Argo CD. 
Aplikacja będzie służyć jako punkt wejścia do uruchamiania zadań w Argo Workflows. 
Użytkownik będzie mógł wysyłać żądania HTTP do aplikacji uruchamiające odpowiednie workflowy za pomocą API do Argo Workflows. 
Każda zmiana w repozytorium przechowującym kod aplikacji będzie automatycznie wykrywana przez Argo CD, które następnie zaktualizuje działającą aplikację na klastrze.


\section{Architektura rozwiązania}
Architektura rozwiązania dla rozważanego studium przypadku widoczna jest na \autoref{fig:env-conf}. Zarówno narzędzia Argo jak i nasza aplikacja (workflow-api) uruchomione są na klastrze Kubernetesa EKS. Argo CD integruje się z Githubem, z którego pobiera zmiany oraz z klastrem w celu aplikowania ich. Workflow-api udostępnia endpointy, na które użytkownicy mogą zlecać obliczenia.

\img{img/suu_env_conf.drawio.png}{Architektura rozwiązania}{fig:env-conf}{0.9}
% \section{Konfiguracja środowiska}


\section{Instalacja narzędzi}
Do zrealizowania projektu niezbędne jest zainstalowanie kilku narzędzi. W zależności od używanego systemu operacyjnego, proces instalacji może się różnić. Poniżej znajdują się linki do instrukcji instalacji dla poszczególnych narzędzi:

\begin{itemize}
    \item \href{https://developer.hashicorp.com/terraform/tutorials/aws-get-started/install-cli}{\textbf{Terraform}}\cite{terraform}
    \item \href{https://kubernetes.io/docs/tasks/tools/}{\textbf{kubectl}}\cite{kubectl}
    \item \href{https://docs.aws.amazon.com/cli/latest/userguide/getting-started-install.html}{\textbf{AWS CLI}}\cite{aws-cli}
    \item \href{https://argo-cd.readthedocs.io/en/stable/cli_installation/}{\textbf{argocd CLI}}
\end{itemize}


\section{Reprodukcja środowiska}
\subsection{Klaster Kubernetes}
Bazowane na \href{https://developer.hashicorp.com/terraform/tutorials/kubernetes/eks}{Provision an EKS cluster (AWS)} oraz \href{https://github.com/hashicorp/learn-terraform-provision-eks-cluster}{Learn Terraform - Provision an EKS Cluster}.
\begin{enumerate}
    \item \code{git clone https://github.com/SUU-Argo/infra}
    \item Umieść konfigurację i dane logowania do AWS w \code{\~{}/.aws/}
    \item W związku z tym, że używamy AWS Academy Learner Lab konieczne jest wykorzystanie istniejącej roli LabRole i podanie jej arn jako zmiennej w terraformie. Pobranie arn: \\
    \code{aws iam get-role -{}-role-name LabRole | grep Arn}
    \item \code{terraform -chdir=infra/terraform init}
    \item \code{terraform -chdir=infra/terraform apply -var="aws\_iam\_role=<your LabRole arn>"}
    \item Sprawdź nazwę klastra: \code{aws eks list-clusters}
    \item Wygeneruj kubeconfig: \code{aws eks update-kubeconfig -{}-name <your cluster name>}
    \item Zweryfikuj połączenie: \code{kubectl get nodes}
\end{enumerate}



\subsection{Argo CD}
Przygotowane w oparciu o \href{https://argo-cd.readthedocs.io/en/stable/getting_started/}{Argo - Getting Started}
\begin{enumerate}
    \item Utwórz namespace: \code{kubectl create namespace argocd}
    \item Zainstaluj ArgoCD: \\
    \codeblock{kubectl -n argocd apply -f https://raw.githubusercontent.com/argoproj/argo-cd/ stable/manifests/install.yaml}
    \item Przekieruj Argo na localhost (zajmuje terminal): \\
    \code{kubectl -n argocd port-forward svc/argocd-server 8080:443}
    \item Uzyskaj dostęp do Argo za pomocą przeglądarki internetowej: \href{https://localhost:8080}{https://localhost:8080}
    \item Uzyskaj hasło: \code{argocd admin initial-password -n argocd}
    \item Zaloguj się: \code{argocd login localhost:8080 -{}-username admin -{}-password <password>}
\end{enumerate}

\subsection{Argo Workflows}
Bazowane na \href{https://argo-workflows.readthedocs.io/en/latest/quick-start/}{Argo Workflows - The workflow engine for Kubernetes - Quick start}
\begin{enumerate}
    \item Stwórz namespace: \code{kubectl create namespace argo}
    \item Zainstaluj Argo Workflows: \\
    \codeblock{kubectl -n argo apply -f "https://github.com/argoproj/argo- \\ 
    workflows/releases/download/v3.5.7/quick-start-minimal.yaml"}
    \item Zaaplikuj łatkę do argo-server (repozytorium infra): \\
    \codeblock{kubectl -n argo patch deployment argo-server -{}-type JSON \\ 
    -{}-patch-file k8s/argo-server-patch.yaml}
    \item Przekieruj argo-server na localhost (zajmuje terminal): \\
    \code{kubectl -n argo port-forward svc/argo-server 2746:2746}
    \item Uzyskaj dostęp do Argo za pomocą przeglądarki internetowej: \href{https://localhost:2746}{https://localhost:2746}
\end{enumerate}

\subsection{workflow-api}
\begin{enumerate}
    \item Ustaw namespace: \code{kubectl config set-context -{}-current -{}-namespace=argocd}
    \item Utwórz aplikację z repozytorium: \\
    \codeblock{argocd app create workflow-api \textbackslash \\
    -{}-repo https://github.com/SUU-Argo/workflow-api.git \textbackslash \\
    -{}-path deploy \textbackslash \\
    -{}-dest-server https://kubernetes.default.svc \textbackslash \\
    -{}-dest-namespace argo}
    \item Wyświetl szczegóły instancji aplikacji: \code{argocd app get workflow-api}, początkowy status to \code{OutOfSync}
    \item Włącz automatyczną synchronizację aplikacji z repozytorium: \\ \code{argocd app set workflow-api -{}-sync-policy automated}
    \item Przekieruj aplikację na localhost (zajmuje terminal): \\ \code{kubectl -n argo port-forward svc/workflow-api 8081:8080}
    \item Zweryfikuj dostępność za pomocą przeglądarki internetowej: \href{http://localhost:8081}{http://localhost:8081}
\end{enumerate}


\newpage
\section{Demo}
\subsection{Argo Workflows}
Zaimplementowana aplikacja workflow-api pozwala poprzez zapytania HTTP tworzyć workflowy w naszym klastrze Kubernetes. Zaimplementowane zostały 3 przykładowe endpointy.

\img{img/swagger.png}{Dostępne endpointy}{fig:workflow-api-endpoints}{0.9}

\begin{description}
    \item{\textbf{/hello}} - przyjmuje parametr \textit{name} i tworzy workflow z workerem drukującym tekst "Hello <name>".
    \item{\textbf{/artifact}} - tworzy workflow składający się z dwóch kroków - pierwszy zapisuje w pliku tekstowym "Hello world", a drugi odczytuje zapisany plik i wypisuje jego zawartość.
    \item{\textbf{/map-reduce}} - przyjmuje parametry \textit{workers} oraz \textit{text} i tworzy workflow, które dzieląc przekazany tekst pomiędzy wskazaną liczbę workerów, zlicza ilość wystąpień każdego słowa, a następnie wyniki przekazywane są do workera, który sumuje wyniki.
\end{description}

Demo opierać się będzie na przedstawieniu działania \textbf{/map-reduce}, w tym celu wykonane zostało zapytanie HTTP w programie Insomnia z przykładowymi parametrami (\autoref{fig:insomnia}). Jako wynik zwrócona została nazwa workflow, którą możemy następnie użyć do odczytania wyniku w Argo Workflows UI.

\img{img/insomnia.png}{Zapytanie HTTP do workflow-api wraz z odpowiedzią}{fig:insomnia}{0.9}

\newpage

W Argo Workflows, zgodnie z przekazanymi parametrami utworzone zostało workflow (\autoref{fig:workers-graph}). W jego skład wchodzi m.in 5 workerów zliczających wystąpienia słów w przydzielonych do workera fragmentach tekstu.

\img{img/workers_graph.png}{Graf workflow}{fig:workers-graph}{0.9}

Otrzymany wynik widać na \autoref{fig:map-reduce-results}.

\img{img/map_reduce_results.png}{Wynik map-reduce}{fig:map-reduce-results}{0.9}

\newpage
\subsection{Argo CD}
Demonstracja możliwości Argo CD opiera się na wprowadzeniu zmian w repozytorium workflow-api. Oczekiwanym zachowaniem jest wprowadzenie tych modyfikacji do klastra w sposób automatyczny.

Początkowy stan aplikacji widoczny przez przeglądarkę internetową (\autoref{fig:root-message-before}).

\img{img/root_message_before.png}{Wiadomość po wykonaniu zapytania przed zmianami}{fig:root-message-before}{0.9}

Wprowadzamy zmiany do repozytorium (\autoref{fig:git-changes}). Ponieważ nasza modyfikacja ogranicza się do kodu aplikacji konieczne jest wypchnięcie nowej wersji obrazu.

\img{img/git_changes.png}{Zmiany w repozytorium}{fig:git-changes}{0.9}

\newpage
Nowy pod został utworzony (\autoref{fig:argocd-pod}).

\img{img/argocd_pod.png}{Widok poda workflow-api w Argo CD UI}{fig:argocd-pod}{0.9}

Deployment jest zsynchronizowany ze stanem w repozytorium (\autoref{fig:argocd-graph}).

\img{img/argocd_graph.png}{Widok deploymentu workflow-api w Argo CD UI}{fig:argocd-graph}{0.9}

\newpage
Zaktualizowana aplikacja zwraca inne dane w głównej ścieżce (\autoref{fig:root-message-after}).

\img{img/root_message_after.png}{Wiadomość po wykonaniu zapytania po zmianach}{fig:root-message-after}{0.9}


\section{Podsumowanie}
Projekt "Argo" miał na celu zademonstrowanie praktycznego zastosowania dwóch kluczowych narzędzi z pakietu Argo: Argo CD oraz Argo Workflows, w środowisku Kubernetes. Argo CD, zainstalowany i skonfigurowany na klastrze Amazon EKS, umożliwił automatyczną synchronizację aplikacji z repozytorium Git, zapewniając ciągłą dostawę aplikacji w modelu GitOps. Argo Workflows pozwolił na tworzenie workflow za pomocą API, co zostało zaprezentowane na przykładzie endpointu map-reduce. \\

Przedstawione demo pokazuje możliwości Argo jako zestawu open source'owych narzędzi wspierających automatyzację, skalowalność i zarządzanie złożonymi zadaniami w nowoczesnych środowiskach chmurowych. Dokumentacja procesu instalacji i konfiguracji, a także stworzone aplikacje demo mogą posłużyć jako materiały referencyjne dla przyszłych implementacji. Projekt udowodnił, że narzędzia Argo są efektywnym rozszerzeniem ekosystemu Kubernetes, umożliwiającym sprawne zarządzanie i wdrażanie aplikacji.


\newpage
% \nocite{*}
\printbibliography[heading=bibintoc,title={Źródła}]

\end{document}
